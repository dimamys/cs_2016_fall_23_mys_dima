\documentclass{book}
\usepackage[english]{babel} 
\usepackage[utf8x]{inputenc} 
\usepackage[T1]{fontenc} 
\usepackage{fancyhdr} 
\usepackage[a4paper,top=2cm,bottom=2cm,left=3cm,right=2cm,marginparwidth=1cm]{geometry} 
\usepackage{xcolor} 
\usepackage{amsmath} 
\usepackage{graphicx} 
\usepackage[colorinlistoftodos]{todonotes} 
\usepackage[colorlinks=true, allcolors=blue]{hyperref} 
\usepackage{setspace} 
\setcounter{chapter}{4}
\setcounter{section}{4}
\setlength{\headheight}{14pt}
\newcounter{pro1}
\setcounter{pro1}{0}
\newcommand{\pro}{\par\addtocounter{pro1}{1}
\textbf{Problem \arabic{chapter}.\arabic{pro1} }\quad}
\setcounter{equation}{37}
\definecolor{light-gray}{rgb}{0.8,0.8,0.8}
\begin{document}
\renewcommand{\headrulewidth}{0pt}
\fancyhf{} 
\pagestyle{fancy}
\chapter{Taking out the big part} \vspace{3cm}
\colorbox{light-gray}{
\begin{minipage}{\textwidth}
5.1 Multiplication using one and few 77\\
5.2 Fractional changes and low-entropy expressions 79\\
5.3 Fractional changes with general exponents     84\\
5.4 Successive approximation: How deep is the well? 91\\
5.5Daunting trigonometric integral 94\\
\textit {5.6 Summary and further problems} \ 97
\end{minipage}
}\vspace{3mm}
In almost every quantitative problem, the analysis simplifies when you follow the proverbial advice of doing first things first. First approximate and understand the most important effect—the big part—then refine your analysis and understanding. This procedure of successive approximationor “taking out the big part” generates meaningful, memorable, and usable expressions. The following examples introduce the related idea of lowentropy expressions (Section 5.2) andanalyze mental multiplication (Section5.1), exponentiation (Section 5.3), quadratic equations (Section 5.4),and a difficult trigonometric integral (Section 5.5).\\
\\
\section{Multiplication using one and few} The first illustration is a method  of mental multiplication suited to rough, back-of-the-envelope estimates. The particular calculation is the storage capacity of a data CD-ROM. A data CD-ROM has the same format and storage capacity as a music  CD, whose capacity can be estimated as the product of three factors: 
\begin{equation} \underbrace{ 1hr \times \frac{3600 s}{1hr}}g \times \underbrace{\frac{4.4 \times 10^4 samples}{1s}} \times \underbrace{2 channels \times \frac{16 bits}{1 sample}.}
\end{equation}
\newpage 
\pagestyle{fancy} 
\renewcommand{\headrulewidth}{0pt} 
\fancyhf{} 
\fancyhead[LE]{\large \textsl {\textbf{78}}} 
\fancyhead[RE]{\large \textsl{5 Taking out the big part}} 
\large\textrm{(In the sample-size factor, the two channels are for stereophonic sound.)}\\
\colorbox{light-gray}{
\begin{minipage}{\textwidth}
\large\textrm{\textbf{\pro Sample rate } \\ 
Look up the Shannon–Nyquist sampling theorem [22], and explain why the samplerate(the rate at which the sound pressure is measured)is roughly 40kHz.\\
\textbf{\pro Bits per sample }\\
Because $2^{16} ∼ 10^5$, a 16-bit sample —-as chosen for the CD format-— requires electronics accurate to roughly 0.001\%. Why didn’t the designers of the CD format choose a much larger sample size, say 32 bits (per channel)? \textbf{\pro Checking units  }\\
Check that all the units in the estimate divide out — except for the desired units of bits.}
\end{minipage}
}\vspace{5mm}
\textrm{Back-of-the-envelope calculations use rough estimates such as the playing time and neglect important factors such as the bits devoted to error detection and correction. In this and many other estimates, multiplication with 3 decimal places of accuracy would be overkill. An approximate analysis needs an approximate method of calculation.}

\vspace{4mm}
\textit{What is the data capacity to within a factor of 2? }
\vspace{4mm}\\
The units (the biggest part!) are bits (Problem 5.3), and the three numerical  factors contribute $3600×4.4×104 ×32$. To estimate the product, split it into a big part and a correction.\\
\textit{The big part}: The most important factor in a back-of-the-envelope product usually comes from the powers of 10, so evaluate this big part first:3600 contributes three powers of 10, $4.4 × 10^4$ contributes four, and 32 contributes one. The eight powers of 10 produce a factor of$10^8$\\
\textit{The correction}: After taking out the big part, the remaining part is a correction factor of 3.6×4.4×3.2. This product too is simplified by taking out its big part. Round each factor to the closest number among three choices: 1, few, or10. The invented number few lies midway between 1 and 10: It is the geometric mean of 1 and 10, so(few)2 = 10 and few ≈ 3. In the product 3.6×4.4×3.2,each factor rounds to few,so $3.6×4.4×3.2 ≈ (few)^3$ or roughly 30.\\
The units, the powers of 10, and the correction factor combine to give 
\begin{equation}capacity ~ 10^8 \times 30 bits = 3 \times 10^9 bits
\end{equation}
\newpage 
\renewcommand{\headrulewidth}{0pt}
\fancyhf{} 
\fancyhead[RO]{\large \textsl {\textbf{79}}} 
\fancyhead[LO]{\large \textsl{5.4 Successive approximation: How deep is the well?}} 
This estimate is within a factor of 2 of the exact product (Problem 5.4), which is itself close to the actual capacity of $5.6×109 bits$.\\
\colorbox{light-gray}{
\begin{minipage}{\textwidth}\textbf{\pro Underestimate or overestimate?}\\ 
Does $3×10^9$ overestimate or underestimate $3600×4.4×10^4×32$? Check your reasoning by computing the exact product.\\
\textbf{\pro More practice }\\
Use the one-or-few method of multiplication to perform the following calculations mentally; then compare the approximate and actual products.\\
\textrm{a.$161 \times 294 \times 280 \times 438.$The actual product is roughly $5.8×109$.\\
b. Earth's surface area $A = 4 \pi R 2$, where the radius is$ R ∼ 6×106$ m. The actual surface area is roughly$ 5.1×1014 m2$.}
\end{minipage}
}\vspace{5mm}
\section{Fractional changes and low-entropy expressions}
Using the one-or-few method for mental multiplication is fast. For example, $3.15×7.21$ quickly becomes$ few × 101 ∼ 30$, which  is  within 50\% of the exact product 22.7115. To get a more accurate estimate, round 3.15 to 3 and 7.21 to 7. Their product 21 is in error by only 8\%. To reduce the error further, one could split $3.15×7.21$ into a big part and an additive correction. This decomposition produces
\begin{equation} (3+0.15)(7+0.21)=\underbrace{3 \times 7}+\underbrace{0.15\time 7 + 3\times 0.21 +0.15 \times 0.21}.
\end{equation}
The approach is sound, but the literal application of taking out the big part produces a messy correction that is hard to remember and understand. Slightly modified, however, taking out the big part provides a clean and intuitive correction. As gravy, developing the improved correction introduces two important street-fighting ideas: fractional changes (Section 5.2.1) and low-entropy expressions (Section 5.2.2). The improved correction will then, as a first of many uses, help us estimate the energy saved by highway speed limits (Section 5.2.3).
\subsection{Fractional changes}
The hygienic alternative to an additive correction is to split the product into a big part and a \textit{ multiplicative} correction:
\newpage 
\pagestyle{fancy} 
\renewcommand{\headrulewidth}{0pt} 
\fancyhf{} 
\fancyhead[LE]{\large \textsl {\textbf{80}}} 
\fancyhead[RE]{\large \textsl{5 Taking out the big part}} 
\begin{equation}
3.15 \times 7.21= \underbrace{3\times 7}\times \underbrace{(1+0.05)\times (1+0.03)}.
\end{equation}
\vspace{5mm}
\textit{Can you find a picture for the correction factor?}\\
The correction factor is the area of a rectangle with width $1 + 0.05$ and height $1 + 0.03$. The rectangle contains one subrectangle for each term in the expansion of $(1+0.05)×(1+0.03)$. Their combined area of roughly $1 + 0.05 + 0.03$ represents an 8\% fractional increase over the big part. The big part is 21, and 8\% of it is1.68, so $3.15×7.21 = 22.68$, which is within 0.14\% of the exact product.\\
\colorbox{light-gray}{
\begin{minipage}{\textwidth}
\large\textrm{\textbf{\pro Picture for the fractional error } \\ 
What is the pictorial explanation for the fractional error of roughly 0.15\%?
\\
\textbf{\pro Try it yourself }\\
Estimate $245×42$ by rounding each factor to a nearby multiple of10,and compare this big part with the exact product. Then draw a rectangle for the correction factor, estimate its area, and correct the big part.
}
\end{minipage}
}
\subsection{ Low-entropy expressions}
The correction to $3.15×7.21$ was complicated as an absolute or additive change but simple as a fractional change. This contrast is general. Using the additive correction, a two-factor product becomes
\begin{equation}(x+ \Delta x)(y+ \Delta y)= xy+\underbrace{x\Delta y+y \Delta x+\Delta x \Delta y}.
\end{equation}
\colorbox{light-gray}{
\begin{minipage}{\textwidth}
\large\textrm{\textbf{\pro Picture for the fractional error } \\ 
What is the pictorial explanation for the fractional error of roughly 0.15\%?
\\
\textbf{\pro Rectangle picture  }\\
Draw a rectangle representing the expansion }
\begin{equation}(x + \Delta x)(y + \Delta y)=xy + x \Delta y + y\Delta x + \Delta x\Delta y.
\end{equation}
\end{minipage}
}\vspace{4mm}
\textrm{When the absolute changes$\Delta x and \delta y$ are small$ (x \gg \Delta x and y \ll \Delta y)$, the correction simplifies to$ x\Delta y+ \Delta x$, but even so it is hard to remember because it has many plausible but incorrect alternatives. For example, it could plausibly contain terms such as$\Delta x \Delta y, x\Delta x, or y \Delta y$. The extent}
\end{document}
# m4_latex
